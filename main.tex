%-------------------------
% Resume in Latex
% Author : Jake Gutierrez
% Based off of: https://github.com/sb2nov/resume
% License : MIT
%------------------------

\documentclass[letterpaper,11pt]{article}

\usepackage[T2A]{fontenc}
\usepackage{latexsym}
\usepackage[empty]{fullpage}
\usepackage{titlesec}
\usepackage{marvosym}
\usepackage[usenames,dvipsnames]{color}
\usepackage{verbatim}
\usepackage{enumitem}
\usepackage[hidelinks]{hyperref}
\usepackage{fancyhdr}
\usepackage[english, russian]{babel}
\usepackage{tabularx}
\input{glyphtounicode}


%----------FONT OPTIONS----------
% sans-serif
% \usepackage[sfdefault]{FiraSans}
% \usepackage[sfdefault]{roboto}
% \usepackage[sfdefault]{noto-sans}
% \usepackage[default]{sourcesanspro}

% serif
% \usepackage{CormorantGaramond}
% \usepackage{charter}


\pagestyle{fancy}
\fancyhf{} % clear all header and footer fields
\fancyfoot{}
\renewcommand{\headrulewidth}{0pt}
\renewcommand{\footrulewidth}{0pt}

% Adjust margins
\addtolength{\oddsidemargin}{-0.5in}
\addtolength{\evensidemargin}{-0.5in}
\addtolength{\textwidth}{1in}
\addtolength{\topmargin}{-.5in}
\addtolength{\textheight}{1.0in}

\urlstyle{same}

\raggedbottom
\raggedright
\setlength{\tabcolsep}{0in}

% Sections formatting
\titleformat{\section}{
  \vspace{-4pt}\scshape\raggedright\large
}{}{0em}{}[\color{black}\titlerule \vspace{-5pt}]

% Ensure that generate pdf is machine readable/ATS parsable
\pdfgentounicode=1

%-------------------------
% Custom commands
\newcommand{\resumeItem}[1]{
  \item\small{
    {#1 \vspace{-2pt}}
  }
}

\newcommand{\resumeSubheading}[5]{
  \vspace{-2pt}\item
    \begin{tabular*}{0.97\textwidth}[t]{l@{\extracolsep{\fill}}r}
      \textbf{#1} & #2 \\
      \textit{\small#3} & \textit{\small #4} \\
      \small#5
    \end{tabular*}\vspace{-7pt}
}

\newcommand{\resumeSubSubheading}[2]{
    \item
    \begin{tabular*}{0.97\textwidth}{l@{\extracolsep{\fill}}r}
      \textit{\small#1} & \textit{\small #2} \\
    \end{tabular*}\vspace{-7pt}
}

\newcommand{\resumeProjectHeading}[4]{
    \item
    \begin{tabular*}{0.97\textwidth}{l@{\extracolsep{\fill}}r}
      \small#1 & #2 \\
      \small#3 & \textit{\small#4} \\
    \end{tabular*}\vspace{-7pt}
}

\newcommand{\resumeSubItem}[1]{\resumeItem{#1}\vspace{-4pt}}

\renewcommand\labelitemii{$\vcenter{\hbox{\tiny$\bullet$}}$}

\newcommand{\resumeSubHeadingListStart}{\begin{itemize}[leftmargin=0.15in, label={}]}
\newcommand{\resumeSubHeadingListEnd}{\end{itemize}}
\newcommand{\resumeItemListStart}{\begin{itemize}}
\newcommand{\resumeItemListEnd}{\end{itemize}\vspace{-5pt}}

%-------------------------------------------
%%%%%%  RESUME STARTS HERE  %%%%%%%%%%%%%%%%%%%%%%%%%%%%


\begin{document}

%----------HEADING----------
% \begin{tabular*}{\textwidth}{l@{\extracolsep{\fill}}r}
%   \textbf{\href{http://sourabhbajaj.com/}{\Large Sourabh Bajaj}} & Email : \href{mailto:sourabh@sourabhbajaj.com}{sourabh@sourabhbajaj.com}\\
%   \href{http://sourabhbajaj.com/}{http://www.sourabhbajaj.com} & Mobile : +1-123-456-7890 \\
% \end{tabular*}

\begin{center}
    \textbf{\Huge \scshape Блещунов Дмитрий} \\ \vspace{1pt}
    \small \href{mailto:bleschunov.dmitry@gmail.com}{\underline{bleschunov.dmitry@gmail.com}} $|$ 
    \href{https://linkedin.com/in/dmitrybleschunov}{\underline{LinkedIn}} $|$
    \href{https://github.com/bleschunov}{\underline{GitHub}}
\end{center}

\textbf{Software Engineer с опытом разработки enterprise—приложения для крупнейшего застройщика РФ.}

%-----------EXPERIENCE-----------
\section{Опыт}
  \resumeSubHeadingListStart

  \resumeSubheading
      {Fullstack разработчик}{Май 2023—Сентябрь 2023}
      {\href{https://t.me/abouthub}{\underline{Hub}}}{Удаленно}
      {LLM-powered ассистент по базам данных для топ–менеджеров в enterprise.\\Получение ответов на вопросы по данным из БД\\с помощью запросов на естественном языке}
      \resumeItemListStart
        \resumeItem{ChatGPT. Интеграция в сервис}
        \resumeItem{Промпт для генерации SQL с использованием уникального языка доменной области компании. Увеличил субъективную точность ответов}
        \resumeItem{Промпт с использованием паттерна chain of thoughts. Сократил количество срабатываний negative false ответов практическо до 0}
        \resumeItem{>15 презентаций генеральному заказчику}
        \resumeItem{Интерфейс чата на React с использованием библиотеки ChakraUI}
        \resumeItem{Схема базы данных для PostgreSQL}
        \resumeItem{API приложения на FastAPI}
      \resumeItemListEnd

    \resumeSubheading
      {Fullstack разработчик}{Сентябрь 2023—Ноябрь 2023}
      {\href{https://t.me/abouthub}{\underline{Hub}}}{Удаленно}
      {LLM-powered ассистент по документам, инструкциям, договорам.\\Для юристов, сотрудников и топ–менеджеров в enterprise.\\Получение ответов на вопросы по документу\\с помощью запросов на естественном языке}
      \resumeItemListStart
        \resumeItem{RAG–система из комбинации двух различных ретриверов. \\Увеличил субъективную точность ответов в 2 раза}
        \resumeItem{Прогресс–бар обработки документа с использованием WebSocket. Улучшил пользовательской опыт}
        \resumeItem{Очередь задач для фоновой обработки документа. Стало возможным моментально отдавать HTTP ответ клиенту}
        \resumeItem{Провёл >15 презентаций генеральному заказчику}
        \resumeItem{API приложения на FastAPI}
      \resumeItemListEnd

    \resumeSubheading
      {Team Lead команды разработки}{Ноябрь 2023—Апрель 2024}
      {\href{https://t.me/abouthub}{\underline{Hub}}}{Удаленно}
      {Семантический маппер строительной номенклатуры\\методом поиска ближайших эмбедингов для застройщиков}
      \resumeItemListStart
        \resumeItem{Классификатор строильной номенклатуры на основе регрессионной модели из scikit-learn, что увеличило точность маппера}
        \resumeItem{Интеграция векторстора ChromaDB для хранения эмбеддингов. Сократил потребление оперативной памяти}
        \resumeItem{E2E тестирование. Устранил несогласованность версий OpenSSL, которая мешала интеграции сервиса с SQL Server заказчика}
        \resumeItem{Провёл >15 презентаций генеральному заказчику}
      \resumeItemListEnd

  \resumeSubheading
      {Fullstack Разработчик}{Февраль 2023—Май 2023}
      {\href{https://vpa.group}{\underline{VPA Group}}}{Удаленно}
      {Платформа футбольной аналитики для РПЛ}

      \resumeItemListStart
        \resumeItem{Разработал >50 правил для подсчёта футбольных метрик на основе временных меток матча}
        \resumeItem{Разработал универсальных класс–репозиторий для создания CRUD операций через наследование от него}
        \resumeItem{Спроектировал >10 REST API эндпоинтов на FastAPI}
        \resumeItem{Разработал >10 компонентов на React}
      \resumeItemListEnd
      

  \resumeSubheading
      {Junior Fullstack Разработчик}{Февраль 2023—Май 2023}
      {\href{https://vpa.group}{\underline{VPA Group}}}{Удаленно}
      {Платформа футбольной аналитики для РПЛ}
      
      \resumeItemListStart
        \resumeItem{Разработал >50 правил для подсчёта футбольных метрик на основе временных меток матча}
        \resumeItem{Разработал универсальных класс–репозиторий для создания CRUD операций через наследование от него}
        \resumeItem{Спроектировал >10 REST API эндпоинтов на FastAPI}
        \resumeItem{Разработал >10 компонентов на React}
      \resumeItemListEnd

  \resumeSubheading
      {Java разработчик}{Декабрь 2022—Январь 2023}
      {CorporationX}{Удаленно}
      {Платформа для поиска контрибьюторов на open-source проект}
      
      \resumeItemListStart
        \resumeItem{Спроектировал >15 эндпоинтов с использованием Spring и Hibernate}
        \resumeItem{Спроектировал >10 таблиц базы данных PostgreSQL}
        \resumeItem{Спроектировал механизм для глобальной обработки исключений}
        \resumeItem{Настроил Spring Security для регистрации и авторизации в приложении}
        \resumeItem{Интегрировал Amazon S3 для хранения пользовательских аватаров в blob хранилище}
        \resumeItem{Интегрировал Amazon SES для отправки e-mail уведомлений пользователям}
        \resumeItem{Настроил логирование для всего проекта}
      \resumeItemListEnd

    \resumeSubheading
      {Фронтенд разработчик}{Июль 2022—Сентябрь 2022}
      {RDXLabs}{Удаленно}
      {Онлайн–платформа для трейдинга}
      
      \resumeItemListStart
        \resumeItem{Разработал компонент для отрисовки графика торгов в личном кабинете}
        \resumeItem{Integrated Formik and Tailwind CSS libraries to increase the productivity of the team}
        \resumeItem{Разработал pixel-perfect лендинг по дизайну из Figma}
      \resumeItemListEnd
      
% -----------Multiple Positions Heading-----------
%    \resumeSubSubheading
%     {Software Engineer I}{Oct 2014 - Sep 2016}
%     \resumeItemListStart
%        \resumeItem{Apache Beam}
%          {Apache Beam is a unified model for defining both batch and streaming data-parallel processing pipelines}
%     \resumeItemListEnd
%    \resumeSubHeadingListEnd
%-------------------------------------------

    % \resumeSubheading
    %   {Information Technology Support Specialist}{Sep. 2018 -- Present}
    %   {Southwestern University}{Georgetown, TX}
    %   \resumeItemListStart
    %     \resumeItem{Communicate with managers to set up campus computers used on campus}
    %     \resumeItem{Assess and troubleshoot computer problems brought by students, faculty and staff}
    %     \resumeItem{Maintain upkeep of computers, classroom equipment, and 200 printers across campus}
    % \resumeItemListEnd

    % \resumeSubheading
    %   {Artificial Intelligence Research Assistant}{May 2019 -- July 2019}
    %   {Southwestern University}{Georgetown, TX}
    %   \resumeItemListStart
    %     \resumeItem{Explored methods to generate video game dungeons based off of \emph{The Legend of Zelda}}
    %     \resumeItem{Developed a game in Java to test the generated dungeons}
    %     \resumeItem{Contributed 50K+ lines of code to an established codebase via Git}
    %     \resumeItem{Conducted  a human subject study to determine which video game dungeon generation technique is enjoyable}
    %     \resumeItem{Wrote an 8-page paper and gave multiple presentations on-campus}
    %     \resumeItem{Presented virtually to the World Conference on Computational Intelligence}
    %   \resumeItemListEnd

  \resumeSubHeadingListEnd


%-----------PROJECTS-----------
\section{Проекты}
    \resumeSubHeadingListStart
      \resumeProjectHeading
          {\textbf{Telegram Helper} $|$ \emph{OOP, asyncio, FastAPI, MongoDB, Telethon}}{Апрель 2024—Настоящее время}
          {Open-source менеджер рассылок в Телеграме}
          {\href{https://github.com/bleschunov/telegram_helper}{\underline{Код на GitHub}}}
          \resumeItemListStart
            \resumeItem{Проектирование асинхронной ООП архитектуры проекта}
            \resumeItem{Составление бизнес–логики сервиса}
            \resumeItem{Декопмпозиция проекта на задачи}
            \resumeItem{Написание кода}
            \resumeItem{Написание юнит–тестов}
            \resumeItem{Привлечение разработчик к работе над проектом}
            \resumeItem{Проектирование бизнес–модели проекта, монетизация}
          \resumeItemListEnd
          
      \resumeProjectHeading
          {\textbf{QrGo} $|$ \emph{Node.js, Express, React, MongoDB, Figma}}{Март 2022—Июнь 2022}
          {Сервис с элементами геймификации для знакомства первокурскников с кампусом университета}
          {\href{https://github.com/bleschunov/qr-go}{\underline{Код на GitHub}}}
        
          \resumeItemListStart
            \resumeItem{Разработал бекенд приложения с использованием Node.js и MongoDB}
            \resumeItem{Разработал дизайн веб–приложения в Figma}
            \resumeItem{Разработал веб–интерфейс на React}
            \resumeItem{Задеплоил демо–версию на Vercel}
          \resumeItemListEnd
    \resumeSubHeadingListEnd



%

%-----------PROGRAMMING SKILLS-----------
\section{Технические навыки}
 \begin{itemize}[leftmargin=0.15in, label={}]
    \small{\item{
     \textbf{Языки программирования}{: Python, JavaScript, TypeScript, SQL, HTML/CSS} \\
     \textbf{Технологии}{: Docker, PostgreSQL, SQL Server, MongoDB, Redis, ChromaDB, Git} \\
     \textbf{Фреймворки \& библиотеки}{: FastAPI, asyncio, Django, Spring, Express, React, Firebase, AWS, Celery, Redis Queue}
    }}
 \end{itemize}

 %-----------EDUCATION-----------
% \section{Education}
%   \resumeSubHeadingListStart
%     \resumeSubheading
%     {Bachelor of Management Information Systems}{October 2022 -- June 2027}
%       {Dokuz Eylül Üniversitesi}{Izmir, Turkey}
%       {The first university which applied the problem-based learning method in Turkey}
%   \resumeSubHeadingListEnd


%-------------------------------------------

% \section{Hobby}
%  \begin{itemize}[leftmargin=0.15in, label={}]
%     \small{\item{
%      Blogging about daily activities on \href{https://twitter.com/bleschunov__}{\underline{Twitter}
%     }}}
%  \end{itemize}
  
\end{document}
