%-------------------------
% Resume in Latex
% Author : Jake Gutierrez
% Based off of: https://github.com/sb2nov/resume
% License : MIT
%------------------------

\documentclass[letterpaper,11pt]{article}

\usepackage[T2A]{fontenc}
\usepackage{latexsym}
\usepackage[empty]{fullpage}
\usepackage{titlesec}
\usepackage{marvosym}
\usepackage[usenames,dvipsnames]{color}
\usepackage{verbatim}
\usepackage{enumitem}
\usepackage[hidelinks]{hyperref}
\usepackage{fancyhdr}
\usepackage[english, russian]{babel}
\usepackage{tabularx}
\input{glyphtounicode}


\pagestyle{fancy}
\fancyhf{} % clear all header and footer fields
\fancyfoot{}
\renewcommand{\headrulewidth}{0pt}
\renewcommand{\footrulewidth}{0pt}

% Adjust margins
\addtolength{\oddsidemargin}{-0.5in}
\addtolength{\evensidemargin}{-0.5in}
\addtolength{\textwidth}{1in}
\addtolength{\topmargin}{-.5in}
\addtolength{\textheight}{1.0in}

\urlstyle{same}

\raggedbottom
\raggedright
\setlength{\tabcolsep}{0in}

% Sections formatting
\titleformat{\section}{
  \vspace{-4pt}\scshape\raggedright\large
}{}{0em}{}[\color{black}\titlerule \vspace{-5pt}]

% Ensure that generate pdf is machine readable/ATS parsable
\pdfgentounicode=1

%-------------------------
% Custom commands
\newcommand{\resumeItem}[1]{
  \item\small{
    {#1 \vspace{-2pt}}
  }
}

\newcommand{\resumeSubheading}[5]{
  \vspace{-2pt}\item
    \begin{tabular*}{0.97\textwidth}[t]{l@{\extracolsep{\fill}}r}
      \textbf{#1} & #2 \\
      \textit{\small#3} & \textit{\small #4} \\
      \small#5
    \end{tabular*}\vspace{-7pt}
}

\newcommand{\resumeSubSubheading}[2]{
    \item
    \begin{tabular*}{0.97\textwidth}{l@{\extracolsep{\fill}}r}
      \textit{\small#1} & \textit{\small #2} \\
    \end{tabular*}\vspace{-7pt}
}

\newcommand{\resumeProjectHeading}[4]{
    \item
    \begin{tabular*}{0.97\textwidth}{l@{\extracolsep{\fill}}r}
      \small#1 & #2 \\
      \small#3 & \textit{\small#4} \\
    \end{tabular*}\vspace{-7pt}
}

\newcommand{\resumeSubItem}[1]{\resumeItem{#1}\vspace{-4pt}}

\renewcommand\labelitemii{$\vcenter{\hbox{\tiny$\bullet$}}$}

\newcommand{\resumeSubHeadingListStart}{\begin{itemize}[leftmargin=0.15in, label={}]}
\newcommand{\resumeSubHeadingListEnd}{\end{itemize}}
\newcommand{\resumeItemListStart}{\begin{itemize}}
\newcommand{\resumeItemListEnd}{\end{itemize}\vspace{-5pt}}

%-------------------------------------------
%%%%%%  RESUME STARTS HERE  %%%%%%%%%%%%%%%%%%%%%%%%%%%%


\begin{document}


\begin{center}
    \textbf{\Huge \scshape Блещунов Дмитрий} \\ \vspace{1pt}
    \small \href{mailto:bleschunov.dmitry@gmail.com}{\underline{bleschunov.dmitry@gmail.com}} $|$ 
    \href{https://t.me/dmitrybleschunov}{\underline{t.me/dmitrybleschunov}} $|$
    \href{https://github.com/bleschunov}{\underline{github.com/bleschunov}}
\end{center}

\textbf{Разработчик с опытом проектирования и создания enterprise—приложений для крупнейшего застройщика РФ}
\newline\textbf{Общий опыт работы разработчиком — 3 года. Мне 21 год. Я номад}

%-----------EXPERIENCE-----------
\section{Опыт}
  \resumeSubHeadingListStart

  \resumeSubheading
      {\textbf{Tech Lead} $|$ \emph{FastAPI, scikit-learn, PostgreSQL, Redis, ChromaDB}}{Ноябрь 2023—Апрель 2024}
      {\href{https://t.me/abouthub}{\underline{Hub}}}{Удаленно}
      {Датастеп Номенклатура. Семантический маппер строительной номенклатуры}
      \resumeItemListStart
        \resumeItem{Разработал классификатор строительной номенклатуры на основе регрессионной модели из scikit-learn, что увеличило точность маппера}
        \resumeItem{Интегрировал векторстор ChromaDB для хранения эмбеддингов, что сократило потребление оперативной памяти}
        \resumeItem{Устранил несогласованность версий OpenSSL на стенде заказчика и в БД заказчика в ходе E2E тестирования, после чего смог интегрировать сервис с MS SQL базой заказчика}
        \resumeItem{Настроил CI/CD, что ускорило разворачивание сервиса на демо–стенде и стенде заказчика}
        \resumeItem{Провёл >15 еженедельных отчётных презентаций генеральному заказчику, что помогло формировать адекватные клиентские ожидания}
        \resumeItem{Спроектировал микросервисное API приложения на FastAPI, что позволило заказчику интегрировать сервис в 1С}
      \resumeItemListEnd

    \resumeSubheading
      {\textbf{Fullstack} $|$ \emph{FastAPI, GPT, React, PostgreSQL, Redis}}{Сентябрь 2023—Ноябрь 2023}
      {\href{https://t.me/abouthub}{\underline{Hub}}}{Удаленно}
      {Датастеп Документы. LLM-powered ассистент по документам и договорам}
      \resumeItemListStart
        \resumeItem{Внедрил в RAG–систему два различных ретривера вместо одного, что увеличило субъективную точность ответов ассистента в 2 раза}
        \resumeItem{Внедрил очередь задач в архитектуру сервиса для фоновой обработки документа, что сократило количество обрывов HTTP соединений по таймауту до 0}
        \resumeItem{Разработал прогресс–бар обработки документа в веб–интерфейсе с использованием WebSocket, что избавило пользователя от ощущения ошибки в процессе загрузки обработки}
        \resumeItem{Провёл >15 еженедельных отчётных презентаций генеральному заказчику, что помогло формировать адекватные клиентские ожидания}
        \resumeItem{Спроектировал микросервисное API приложения на FastAPI, что позволило другим отделам компании–заказчика использовать наш сервис в своих клиентах}
      \resumeItemListEnd

\resumeSubheading
  {\textbf{Fullstack} $|$ \emph{FastAPI, GPT, React, PostgreSQL}}{Май 2023—Сентябрь 2023}
  {\href{https://t.me/abouthub}{\underline{Hub}}}{Удаленно}
  {Датастеп Базы Данных. LLM-powered ассистент по базам данных}
  \resumeItemListStart
    \resumeItem{Разработал систему парсинга ответов от OpenAI GPT API на основе functional calling, что сократило количество ошибок из–за некорректного ответа от LLM в несколько раз}
    \resumeItem{Разработал промпт для генерации SQL с использованием уникального языка доменной области компании, что увеличило субъективную точность ответов}
    \resumeItem{Разработал промпт с использованием паттерна chain of thoughts, что сократило количество срабатываний negative false ответов практически до 0}
    \resumeItem{Провёл >15 еженедельных отчётных презентаций генеральному заказчику, что помогло формировать адекватные клиентские ожидания}
    \resumeItem{Разработал прототип веб–интерфейса за сутки с помощью Python–библиотеки Streamlit, что ускорило разработку интерфейса на React с использованием библиотеки ChakraUI}
    \resumeItem{Спроектировали multi tenants архитектуру базы данных PostgreSQL, что позволило сделать приложение cloud-based}
    \resumeItem{Спроектировал микросервисное API приложения на FastAPI, что позволило другим отделам компании–заказчика использовать наш сервис в своих клиентах}
  \resumeItemListEnd

  \resumeSubheading
      {Fullstack Разработчик}{Февраль 2023—Май 2023}
      {\href{https://vpa.group}{\underline{VPA Group}}}{Удаленно}
      {Платформа футбольной аналитики для РПЛ}
      \resumeItemListStart
        \resumeItem{Разработал >25 SQL правил для подсчёта футбольных метрик на основе временных меток матча, что значительно расширило функционал приложения}
        \resumeItem{Разработал универсальный класс–репозиторий для создания CRUD операций через наследование от него, что повысило скорость разработки}
      \resumeItemListEnd

  \resumeSubHeadingListEnd


%-----------PROJECTS-----------
\section{Проекты}
    \resumeSubHeadingListStart
      \resumeProjectHeading
          {\textbf{Telegram Helper} $|$ \emph{OOP, asyncio, FastAPI, MongoDB, Telethon}}{Апрель 2024—Настоящее время}
          {Open-source менеджер рассылок в Телеграме}
          {\href{https://github.com/bleschunov/telegram_helper}{\underline{Код на GitHub}}}
          \resumeItemListStart
            \resumeItem{Проектирую, разрабатываю и продвигаю проект}
          \resumeItemListEnd
          
      \resumeProjectHeading
          {\textbf{QrGo} $|$ \emph{Node.js, Express, React, MongoDB, Figma}}{Март 2022—Июнь 2022}
          {Сервис для знакомства первокурсников с кампусом университета}
          {\href{https://readymag.website/u518425071/4178303/}{\underline{Презентация и код}}}
          \resumeItemListStart
            \resumeItem{Сделал дизайн в Figma, разработал бекенд на Node.js и MongoDB, а веб–интерфейс — на React, задеплоил демо–версию на Vercel}
          \resumeItemListEnd
    \resumeSubHeadingListEnd


%-----------PROGRAMMING SKILLS-----------
\section{Технические навыки}
 \begin{itemize}[leftmargin=0.15in, label={}]
    \small{\item{
     \textbf{Языки программирования}{: Python, SQL, JavaScript (TypeScript), HTML/CSS} \\
     \textbf{Технологии}{: Git, Docker, PostgreSQL, SQL Server, MongoDB, Redis, ChromaDB} \\
     \textbf{Фреймворки\&библиотеки}{: FastAPI, asyncio, Django, Express, React, Firebase, AWS, Celery, Redis Queue, Pandas}
    }}
 \end{itemize}

 \end{document}
